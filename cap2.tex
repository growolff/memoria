\chapter{Marco teórico}

En este capitulo, se introducen conceptos básicos sobre robots y sus componentes.

La primera sección describe al robot industrial utilizado y sus componentes de hardware.

Finalmente se hace una breve descripción de distintos componentes de software utilizados.


\section{Manipulador robótico}

\subsection{Tipos de manipuladores}

\subsection{Características Scorbot ER VII}

\section{Conceptos de cinemática}

\subsection{Cinemática directa}

\subsection{Cinemática inversa}



Control de motores DC
Estructura
Sistemas electrónicos involucrados
Concepto de tiempo real
Buses de campo
Interfaces hápticas
Phantom Omni
ROS
Aplicaciones como robot picarocas


\section{Componentes de software}

\subsection{ROS}

Hola esto es una cita a ROS \cite{quigley2009}.



Esto es una cita a TF \cite{foote2013}.

cita a Orocos \cite{orocos2001}

\cite{craig1989}

\cite{gier2015}

\cite{diankov2010}

